\documentclass[../draft_1.tex]{subfiles}

\begin{document}

\chapter{Prologue}

In this thesis we would like to discuss, develop and present adaptive multigrid solvers using space-time discretisations solving parabolic reaction diffusion equations with a potentially nonlinear forcing term. They present a broad class of partial differential equations that can be written in the following form
\begin{ceqn}
\begin{equation}
u_t - \nabla \cdot ( D(x) \nabla u) = f(u)
\end{equation}
\end{ceqn}
for some $u = u(x,t)$ in a space domain over a time interval in addition to a set of boundary conditions. We will postpone more rigorous definitions to the following chapters and for now simply assume the problem to be well posed. These type of equations are used to describe a variety physical phenomena, among which there are the change of one or more chemical substances into another over time as well as the propagation of wavefronts which will be discussed in much more detail later \cite{zegeling2004adaptive} or the modeling of the development of animal populations in biology \cite{cosner2008reaction}. In its easiest form we have a zero source term, that is $f = 0$, this describes a simple heat equation, that is the variation of temperature in a particular region over time starting from a set of initial conditions which will eventually reach an equilibrium state.
\smallskip
\\
Due to their great applicability and importance there has been and continuous being an extensive interest in efficiently solving this type of equations [source]. And while there is a wide variety of approaches used in order to find a numerical approximation to a solution of (1.1) we will consider the following; a space-time parallel discretisation using a least squares finite element approach whose linearisations will in turn give rise to sparse, symmetric linear systems of equations which we will then solve using different adaptive multigrid methods. This suffices in the linear case. That is, if we have a nonlinear source or reaction term $f$, we additionally require an outer iteration succuessively solving linearisations of the above type, which are intended to converge to the final solution. While this composition of methodologies may at first seem like a rather complicated construction we will justify in the subsequent paragraphs and chapters why we consider these particular choices to be advantegeous and will also be presenting numerical results. Having said this we can say that this is a fairly unique construction that has to our knowledge not been studied in this context and will therefore require further investigations before drawing any final conclusions on its utility.
\smallskip
\\
We are interested in large scale systems, consisting of coupled equations with a great number of degrees of freedom. Assuming that we would like to model real life phenomena this is a realistic assumption, considering the complexity of the above mentioned applications which usually require a very high accuracy in time and usually several space dimensions, and are often very sensitive to these parameters. A more precise account on the wide-ranged scales that can be involved are presented in the following chapter. Hence one has to assume that the final linear systems we strive to solve for are very large. Therefore we are searching for an accurate, efficient and robust way to do so. When trying to numerically approximate the solution of a partial differential equation there is no unique way to do so and hence one encounters many choices that have to be made. One of them entails the way we deal with the evolution of the equation over time. Traditionally this is done by recursively computing an approximation for all space elements or nodes at a certain time $t_n$ and then using those results or even preceding ones to compute the approximate solution at the next time step $t_{n+1}$. This seems like the natural way to perform these operations, mainly because this is how we move through time, sequentially, and second because the solution at a given time usually depends on if it is not fully determined by the previous ones, so why not use that information. Computationally this has a large drawback, especially when dealing with large time scales, in order to compute a solution at time $t_n$, one has to wait until all other computations on the previous time steps are completed. Especially in a time where the clockspeed on computers does not increase anymore but the only way to achieve further speed up is through the usage of more processors parallelisation is key. Thus it seems favourable to  consider a space-time discretisation which can then hopefully be solved in a parallel set up if we choose the remaining strategies to solution appropriately. A further discussion as well as references to space-time discretisations can be found in section 3.1. 
\smallskip
\\ 
A least squares finite element method entails the construction of an optimisation problem whose solution coincides with the solution of the differential equation. Instead of solving the original problem we now apply a finite element approach in space-time to solve the auxiliary problem. This particular choice was made due to the fact that least squares formulations give rise to symmetric systems of equations, and when using a Galerkin approach for the discretisation the arising system will be sparse and in the linear case even positive definite. In section 3.3 we will be discussing the most important properties of finite element methods in general, and the method of least squares in this setting will be introduced in section 3.4. We especially wanted the constituting linear system of equations to be symmetric for reasons which will become more evident throughout this thesis and involve the development a particular adaptive multigrid method that makes use of inherent properties of certain reaction-diffusion equations.
\smallskip
\\
Multigrid methods in general represent a powerful class of iterative solvers to approximate the solution of large, sparse, symmetric, positive definite linear systems of equations, usually arising from the discretisation of differential equations [source]. 
They grant us the assembly of a system involving all time steps at once while allowing for parallelisation. There are different varieties of how to construct the multilevel spaces, how to do so well for parabolic equations will be the matter of discussion in chapter 5, especially in the later sections. 
\smallskip
\\
A forcing term $f$ that depends on the solution itself, that is $f = f(u)$ and that might in addition be non-linear can immensely complicate the process of solving this type of equation. The system of equations arising from the linear case including well-defined boundary conditions has a unique solution, and the auxiliary minimisation problem has a single global minimiser, however this is not true in the nonlinear case, which might have an arbitrary number of local minima.  Depending on the solution landscape one might therefore require an initial guess that is already very close to the global solution to ensure convergence. This is especially true for a space-time parallel set up where an initial guess already requires values for all time steps. The non-linear solvers that were employed in the implementation section here, see chapter 6, are a trust region method and a damped Newton method, whose theoretical introduction can be found in section 3.2. Below we can see a schematic overview of how the beforementioned core concepts are tied together in order to give rise to a comprehensive \textit{solver}. 
\smallskip
\\
Hence these are the main ingredients that we will tie together in this thesis in order to develop an efficient, robust and accurate solver to tackle problems of type (1.1). Each of them will be introduced more thoroughly in chapter 3, where we will also explain the particular choice for each of them in more detail, attempting to make use of their favourable attributes while trying to avoid the pitfalls. In chapter 4 we derive a proper problem formulation, which we will then discretise in order to derive linear systems of equations to be solved iteratively. Afterwards we introduce multrigrid methods in chapter 5, especially discussing the particularities that arise due to the construction presented in chapter 4. Chapter 6 then contains the numerical results we obtained for various test cases and discusses certain behaviors we observed during our work which will then be followed by a conclusion and an outlook in chapter 7. In order to not get lost in one of the many intermediate steps leading to our overall set up there will be a short paragraph at the beginning of each chapter roughly describing the main ideas and intentions of what is about to follow.

\begin{framed}
	\underline{\textbf{Overview of the different Steps towards an Approximate Solution}} 
	
	\begin{enumerate}
		\item  Reformulate (1.1) as a mimimisation problem $J$ whose solution coincides with the one of the original equation. 
		\item Discretise the problem using a space-time finite element approach
		\item Derive a non-linear iteration scheme (e.g. Newton or Trust Region method) where we solve a linearisation of the problem using the current iterative solution in each step
		\item  Solve the arising linear system of equations using an adaptive multigrid method
		\item Repeat step 4 with the updated solution each time until some stopping criterion is met 
	\end{enumerate}	
\end{framed}

In order to obtain a meaningful solution $u$ we need a number of properties to be fulfilled. In each nonlinear iteration step the multigrid solver has to converge to the solution of the linearised least squares minimisation problem which mimics the corresponding linearisation of the original partial differential equation. In the outer iteration we need the nonlinear iteration scheme to converge to the minimum of our non-linear functional whose solution as mentioned above is supposed to correspond to the solution of the original problem. However we are not ensured global convergence since the problem is in general non convex.
\bigskip
\\
There is one particular application that originally motivated the construction of such a solver. The propagation of electric signals in human heart tissue who can be modeled using the so-called monodomain equations which are also a reaction-diffusion system. The contraction of our heart is governed by an electric impulse whose charge distribution travels as a wavefront through our cell tissue. One of the main difficulties that arises in this setting is the fact that there are relatively large spatial areas as well as time spans where we exhibit no changes in the current potential and then very drastic changes occur rapidly. And while one therefore requires a very precise approximation in areas of the traveling wavefront which goes hand in hand with an highly increased computational cost, this is a waste of resources in times and areas where there is almost no change. 

Therefore in this thesis we are aiming for a better understanding of the versality of space-time least squares finite element approaches in general and in combination with multigrid methods. But then an additional focus will be given to the construction of a particular algebraic multigrid method that takes intrinsic properties related to the monodomain equation into account, developing an equally accurate but more efficient way through an adapted coarse grid construction. To allow for a better overall understanding of the processes involved in this particular application the following chapter will give a brief insight into the functioning of the human heart, the transmission of electric potential through tissue, the different charge distribution within or between cells or cellular structures and how this can be turned into a mathematical model. 

\end{document}
